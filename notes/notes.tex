\documentclass[11pt]{article}

% Order is important in package loading
% Load math packages and set margins
\usepackage[margin=1in]{geometry} 
\usepackage{amsmath, amsthm, amssymb, amsfonts}
\usepackage{fancyhdr, color, comment, graphicx, environ, bbm}
\usepackage{multirow}
\usepackage{algorithm}
\usepackage{algpseudocode}
\renewcommand{\algorithmicensure}{\textbf{Output:}}

\usepackage{capt-of}
\usepackage{longtable}
\usepackage{parskip}

% to squeeze more space in
\usepackage[compact]{titlesec}
\titlespacing{\section}{0pt}{2ex}{1ex}
\titlespacing{\subsection}{0pt}{2ex}{1ex}
\titlespacing{\subsubsection}{0pt}{2ex}{1ex}
    
% Libertine Font
\usepackage{libertine}
\usepackage{mweights}
\usepackage[bigdelims,cmintegrals,libertine,vvarbb]{newtxmath}
\usepackage{zi4}

\usepackage[T1]{fontenc}
\usepackage{hyphenat}
\usepackage{blkarray}
\usepackage{bbding}
\usepackage[labelfont=bf]{caption}
\usepackage{subcaption}
\usepackage{xspace}
\usepackage{xcolor}
\usepackage[shortlabels]{enumitem}
\usepackage{titling}
\usepackage{varwidth}
\usepackage{booktabs}
\usepackage{import}
\usepackage[subpreambles=false]{standalone}
\usepackage{placeins}
\usepackage{svg}

\usepackage{tikz}
\usepackage{tikz-qtree}
\usetikzlibrary{matrix, positioning, automata, shapes.multipart}
\usetikzlibrary{arrows.meta}
\usetikzlibrary{shapes}
\usepackage{pgfplots}
\pgfplotsset{compat=newest}
\usepgfplotslibrary{colorbrewer}

\usepackage[framemethod=TikZ]{mdframed}
\mdfsetup{
    linewidth=1.5pt,
    innertopmargin=\dimexpr5pt-\topskip\relax
}

% Load hyperref last
\PassOptionsToPackage{hyphens}{url}
\usepackage{hyperref}
\usepackage{cleveref}

\hypersetup{
  colorlinks=true,
  linkcolor=blue,
  filecolor=magenta,      
  urlcolor=blue,
  citecolor=blue
}

\renewcommand{\algorithmicensure}{\textbf{Output:}}
\usepackage{thmtools}
\usepackage{thm-restate}

\newtheorem{theorem}{Theorem}
\newtheorem{conjecture}{Conjecture}
\newtheorem{lemma}{Lemma}
\newtheorem{corollary}{Corollary}
\newtheorem{proposition}{Proposition}
\newtheorem{observation}{Observation}
\newtheorem{definition}{Definition}
\newtheorem{problem}{Problem}
\newtheorem{example}{Example}

\renewcommand{\qedsymbol}{$\blacksquare$}


\newcommand{\henri}[1]{\textcolor{blue}{[#1]}}
\newcommand{\tree}{\mathcal{T}}

\usepackage{todonotes}

\title{A linear time algorithm for VAFPP projection}
\author{Henri Schmidt}
\date{\today}

\begin{document}
\maketitle

\section{Model for tumor deconvolution}

\begin{definition}
  A rooted tree $\,\tree$ on $n$ vertices is an $n$-clonal tree 
  for a mutation set $[n] = \{1, \ldots, n\}$ if each edge is
  labeled by exactly one mutation in $[n]$.
\end{definition}

Let $F$ be an $n$-by-$m$ matrix of frequencies measured on
a set of $m$ mutations across a set of $n$ samples. Given the
matrix $F$, the variant allele frequency projection problem (VAFPP) is to

\begin{problem}
  \label{prob:vafpp}
  Given a frequency matrix $F$ and a clonal matrix $B$, the 
  \emph{variant allele frequency $p$-projection problem} ($p$-VAFPP) is to
  find a usage matrix $U$ such that 
  \begin{equation}
    \sum_{i=1}^m\lVert F_i - (UB)_i \rVert_p
  \end{equation}
  is minimized.
\end{problem}

First, notice that it suffices to consider the case where there is only a
single sample, since the objective is separable with respect to the samples. 
That is, we can assume $F$ is a row vector, which we denote as $f^T$, and the
goal is to find a usage vector $u^T$ such that $\lVert f^T - u^TB \rVert_p$ is minimized.

Here, we will consider the case where $p = 1$, since it has not yet been 
studied in the literature and is more robust to outliers. 
The case where $p = 2$ is the well-known case studied by \cite{jia_efficient_2018},
and they derive an efficient $\mathcal{O}(mn^2)$ time algorithm to solve the
$2$-VAFPP problem.

We start by writing out a linear programming formulation of the $1$-VAFPP problem.
Let $f^T$ be a row vector of frequencies, and let $u^T$ be a row vector of usages.
Let $B$ be an $n$-by-$n$ clonal matrix. Then, the $1$-VAFPP problem is equivalent
to the following linear program.

\begin{mdframed}
\begin{align}
  \max_{u \geq 0, z \geq 0} &-\sum_{i=1}^n z_i \nonumber \\
  \text{subject to }\quad z_i &\geq f_i - \sum_{j=1}^n u_j B_{ji} \quad\text{ for all } i \in [n] \label{eq:constr1} \\
  z_i &\geq \sum_{j=1}^n u_j B_{ji} - f_i \quad\text{ for all } i \in [n] \label{eq:constr2} \\
  1 &\geq \sum_{i=1}^n u_i \label{eq:constr3}
\end{align}
\end{mdframed}

Then, we can write out the dual problem by associating
a dual variable $\alpha_i$ with the constraint in \eqref{eq:constr1},
a dual variable $\beta_i$ with the constraint in \eqref{eq:constr2},
and a dual variable $\gamma$ with the constraint in \eqref{eq:constr3}.
Then, the dual linear program is as follows.

\begin{mdframed}
  \begin{align}
    \min_{\alpha \geq 0, \beta \geq 0, \gamma \geq 0} \gamma + \sum_{i=1}^n f_i(\beta_i - \alpha_i) \nonumber \\
    \text{subject to } \quad 
    \sum_{j=1}^n B_{ij}(\beta_j - \alpha_j) + \gamma &\geq 0 \quad\text{ for all } i \in [n] \label{eq:dualconstr0} \\
    \alpha_i + \beta_i &\leq 1 \quad\text{ for all } i \in [n] \label{eq:dualconstr1}
  \end{align}
\end{mdframed}

We can perform a change of variables by setting $\lambda_i = \beta_i - \alpha_i$.
Since $\alpha_i$ and $\beta_i$ are non-negative and their sum is bounded by $1$, 
$\lambda_i \in [-1, 1]$. Then, writing the constraints in matrix form and 
using a slack variable to remove the inequality constraint, we have
the following equivalent, dual linear program.

\begin{mdframed}
  \begin{align}
    \min_{\gamma \geq 0, \psi \geq 0} \gamma + f^T\lambda \\
    \text{subject to } \quad 
    B\lambda &= \psi - \gamma\mathbbm{1} \label{eq:dualconstr0} \\
    \lambda_i &\in [-1, 1] \quad\text{ for all } i \in [n]
  \end{align}
\end{mdframed}

We now make use of the following lemma.

\begin{lemma}
  \label{thm:main}
  Let $B$ be an $n$-by-$n$ clonal matrix and $A$ the corresponding ancestor-child matrix,
  where $A_{i,j} = 1$ if $j$ is a parent of $i$ and is otherwise 0. Then, 
  $$B = (I - A)^{-1} \quad\text{and}\quad [(I - A)v]_i = \begin{cases}
    v_i - v_{\text{parent}(i)} &\text{ if } i \neq \text{root}, \\
    v_i &\text{ otherwise}.
  \end{cases}$$
  where $\text{parent}(i)$ is the parent of vertex $i$ in the tree corresponding to $B$.
\end{lemma}

Applying the above lemma and noting that $(\psi_i -\gamma) - (\psi_j - \gamma) = \psi_i - \psi_j$,
we obtain:
$$\lambda_i = \left[(I - A)^{-1}(\psi - \gamma\mathbbm{1})\right]_i = \begin{cases}
  \psi_i - \psi_{\text{parent}(i)} &\text{ if } i \neq \text{root}, \\
  \psi_i - \gamma &\text{ otherwise}.
\end{cases}
$$

Finally, noting that $\lambda_i \in [-1, 1]$ and $\psi_i, \gamma$ non-negative implies
$\psi_i, \gamma \in [0, 1]$, we can remove the variable $\lambda$. Then,
re-writing the objective as a linear function of $\gamma$ and $\psi$, we have 
the following equivalent, dual linear program.

\begin{mdframed}
  \begin{align}
    \min \qquad&\gamma(1-f_{\text{root}}) + \sum_{i=1}^n \psi_i\left(f_i - \sum_{j \in \text{child}(i)} f_j\right) \\
    \text{subject to } \quad &\psi_i, \gamma \in [0, 1]
  \end{align}
\end{mdframed}

Notice that this linear program is trivial to solve, by setting
$$\gamma = 0 \text{ and } \psi_i = \begin{cases}
  0 &\text{ if } f_i \geq \sum_{j \in \text{child}(i)} f_j, \\
  1 &\text{ otherwise}.
\end{cases}$$
which takes objective value $0$ if and only $f$ satisfies the sum condition,
providing another proof of the sufficiency of this condition.

\begin{theorem}
  Given a frequency vector $f \in \mathbb{R}^n$ and a clonal matrix $B \in \mathbb{R}^{n \times n}$,
  the minimum of 
  $$\lVert f^T - u^TB \rVert_1$$
  over all usage vectors $u \in \mathbb{R}^n$ is equal to
  $$\sum_{i=1}^n \max\left\{0, \sum_{j \in \text{child}(i)} f_j - f_i\right\},$$
  where $\text{child}(i)$ is the set of children of vertex $i$ in the tree corresponding to $B$.
\newpage
\bibliographystyle{plain}
\bibliography{references}

\end{document}


